\section{Conclusion}\label{sec:con}
%In this paper, we analyze some problems of currently quantization methods, such as the strong subjectivity of researchers, based on experience guidance rather than theory and so on. Based on the existing problems, we propose ULQ, which is based on the assumption that the weights satisfy the normal distribution. We fully analyze the best solution that ULQ can theoretically achieve. Theoretical analysis shows that the optimal solution of ULQ has no straightforward solutions. But we novelly solve this problem by normalizing weights to a standard normal distribution. By limiting the conditions, we can solve this problem by enumerating all the data. Finally, in our experimental part, we verify the correctness of our theoretical analysis by simulating a complete normal distribution, and verify our assumption is reasonable through analysing the actual weights of the classical model, that is, most of the model weights satisfy the normal distribution.
In this paper, we realize that making assumptions about raw data is crucial. First, it limits the problem to a certain extent, rather than trying to solve all problems in one way. Second, it allows our method to be analyzed, and we can try to solve the problem using the tools in mathematical or machine learning,. This means that for differently distributed data, we should consider different methods to achieve lower data loss. 
In addition, we also find that the flexibility of quantization is the key to achieving precision and bit width trade-off. The fixed bit width based quantization method is difficult to apply to different DNN models, even if the weights of these models are all equal to the same distribution (such as the normal distribution).
%\emph{Future works: }Our future works will be the applying of ULQ to model training.
\section{Acknowledgement}
This work is partially supported by the National Natural Science Foundation (61872200, 11801284), the Natural Science Foundation of Tianjin (17JCQNJC00300), the Science and Technology Commission of Tianjin Binhai New Area (BHXQKJXM-PT-ZJSHJ-2017005) and the National Key Research and Development Program of China (2016YFC0400709).